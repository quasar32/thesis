%%%%%%%%%%%%%%%%%%%%%%%%%%%%
%  Template code for the Undergraduate Research Scholars thesis program starting, updated by Undergraduate Research Scholars program staff. Version 6.0. Last Updated: Fall 2024
%  Modified by Tawfik Hussein from the template code for TAMU Theses and Dissertations starting Spring 2018, authored by Sean Zachary Roberson. Version 3.17.09.
%
%%%%%%%%%%%%%%%%%%%%%%%%%%%%
%%%%%%%%%%%%%%%%%%%%%%%%%%%%
% ABSTRACT (REQUIRED PAGE - DO NOT REMOVE)
%%%%%%%%%%%%%%%%%%%%%%%%%%%%

%_________(0)_________
% % Do not modify. This is the required page heading. 

\chapter*{\large\bf ABSTRACT}

%_________(1)_________
% Do not modify. This adds the Abstract section to the Table of Contents.

\addcontentsline{toc}{chapter}{ABSTRACT} 

%_________(2)_________
% Do not modify. This controls the vertical spacing between the word "Abstract" and the thesis title.

\vspace{-1em}

%_________(3)_________
% Do not modify. This section single-spaces and centers the thesis title.

\begin{singlespace}

\begin{center}

%_________(4)_________
% Modifications Needed!
% Enter your thesis title in Title Case exactly as it appears on the Title page.

Parallelizing Collisions Between Spheres With Edge Coloring \\
\vspace{3em}

%_________(5)_________
% Modifications Needed!
% Enter your name or team names exactly as they appear in other sections (spelling, order, superscripts, etc.). Enter department affiliations. Note that your major may be different than your department name.

James Fontenot \\
Department of Computer Science and Engineering \\
Texas A\&M University

\vspace{3em}

%_________(6)_________
% Modifications Needed!
% Enter the name(s) and department(s) of your advisor(s). Don't forget to include "Dr." if they have a PhD. If they don't have a PhD, use their post-nominal letters. 
% The template includes two advisor slots. If you only have one, delete the secondary advisor slot. However, if you have two or more advisors, be sure to include them on the Title page exactly as you list them here. If both advisors are in the same department, you can place their names on the same line. However, if they belong to different departments, give them their own sections. If you have more than two advisors, copy and paste one of the sections below text below and modify accordingly.

Faculty Research Advisor: Dr. Shinjiro Sueda \\
Department of Computer Science and Engineering \\
Texas A\&M University

\vspace{3em}
\end{center}
\end{singlespace}

%_________(7)_________
% Do not modify. This section starts your thesis at page 1 on the Abstract page (required).

\pagestyle{plain}
\pagenumbering{arabic}
\setcounter{page}{1}

%_________(8)_________
% Modifications Needed!
% Use the Double /indent command to properly indent all paragraphs. Enter your abstract text.

\indent \indent Computers are used to simulate systems with many particles, such as fluid behavior in fluid dynamics and intermolecular interactions in molecular dynamics. 
The Gauss-Seidel method may be used to accurately simulate such systems in a single-threaded environment, but since these simulations involve so many particles they need to be parallelized in order to be performant. 
The Jacobi method may be used to make a simulation parallelizable, but suffers multiples issues such as higher memory usage, instability, and slower convergence. 
This paper focuses on instead making resolving collisions with
Gauss-Seidel parallelizable using edge coloring.
The space of the simulation is subdivided into a grid and a graph is formed over the connectivity of the grid. 
All edges of the grid are colored in such a way that indicent edges do not share a color.
For each color, the associated set of edges are evenly split between threads and collisions
are resolved within each edge.
This preserves the fast convergence of Gauss-Seidel while requiring virtually no increase memory usage over its single threaded counterpart. 
When tested, a four-threaded version of the code increased performance by over 200\%. 
In the future, this method can be ported to the GPU to further improve performance.


\pagebreak{}
